\section{Propagation in ice and air}
\label{sec:propagation}



The electric field is propagated to the payload using a standard ray tracing algorithm.
The electric field magnitude at the payload position (before applying
the ANITA instrument response) is calculated as follows:
\begin{equation}
 \mathcal{E}_{\perp,||} = \mathcal{E}^{(\mathrm{@ 1m})}  
 \mathrm{e}^{(-d_{\mathrm{ice}}\ /\ \ell_{\mathrm{attn}})}
 \frac{t_{\perp,||}}{d_{\mathrm{ice}} + d_{\mathrm{air}}}
 \ F(\theta_{\mathrm{view}}-\theta_{\mathrm{Ch}}) \,.
\end{equation}
The first factor after $\mathcal{E}^{(\mathrm{@ 1m})}$ accounts for propagation in ice, with $d_{\mathrm{ice}}$ being the path length and $\ell_{\mathrm{attn}}$ the attenuation length in ice. 
The second factor accounts for the refraction from ice to air, using
the specular Fresnel coefficients, $t_{||}$ and $t_{\perp}$, for the parallel and perpendicular components of the normalized transmitted electric field vector at the ice-air interface with respect to the local surface normal.
The surface normal includes a Gaussian 1.2\% direction perturbation to account for surface slope effects.
Finally, $F(\theta_{\mathrm{view}}-\theta_{\mathrm{Ch}})$ is a geometrical attenuation factor of the field strength in air resulting from viewing the Cherenkov emission at an angle $\theta_{\mathrm{view}}$ different from the Cherenkov angle $\theta_{\mathrm{Ch}}$.
The functional form of $F$ is taken to be Gaussian and is set to zero for ($\theta_{\mathrm{view}}-\theta_{\mathrm{Ch}}$) > 20 $\Delta \theta_{\mathrm{Ch}}$, where $\Delta \theta_{\mathrm{Ch}}$ is the width of the Cherenkov cone. 


Cherenkov radiation is radially polarized, and the in-ice polarization vector of the Cherenkov radiation is radially outward in the plane formed by
the neutrino velocity vector and the Cherenkov propagation vector.
Attenuation lengths for radio in Antarctica are based on measurements performed at the Ross Ice Shelf and the South Pole~\cite{smex}.
The index of refraction is taken as 1.79 for deep ice and 1.325 at the surface. A model for the firn, a layer of packed snow above the ice, based on data taken by the RICE Collaboration~\cite{PhysRevD.73.082002} is used at depths shallower than 150\,m.

Surface roughness acts to allow different portions of the ice surface to contribute transmitted power to the payload for a given event, because it allows new scattering geometries at the air-ice interface.
The University of Hawaii ANITA group developed a simulation program, used for the ANITA-I and ANITA-II flights, which included surface roughness effects. They found that ice surface roughness contributed to an increase in acceptance of roughly a factor of 50\% at low energy (close to $10^{18}$\,eV) and 40\% at higher energy.
\icemc currently does not include ice surface roughness effects, and hence provides a conservative estimate of the sensitivity of the ANITA flights. Future versions of \icemc will include these effects.



%%% Local Variables:
%%% mode: latex
%%% TeX-master: "icemc.tex"
%%% End:
