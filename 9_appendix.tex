\appendix

\section{Obtaining and using \icemc}
\icemc can be obtained from the GitHub repository \url{https://github.com/anitaNeutrino/icemc}.
The only mandatory requirement for the compilation of \icemc is a ROOT~\cite{brun1997root} installation.
The compilation can be executed via the Makefile or using the CMake structure.

To run \icemc one needs to define two environment variables and add them to the global path:
\begin{itemize}
    \item  \texttt{ICEMC\_SRC\_DIR} should point to the directory where the source code is saved;
    \item  \texttt{ICEMC\_BUILD\_DIR} should point to the directory where the executable programs are.
\end{itemize}

To run icemc one can simply do:
\begin{verbatim}
./icemc -i {inputFile} -o {outputDirectory} -r {runNumber}
-n {numberOfNeutrinos} -t {triggerThreshold} -e {energyExponent}
\end{verbatim}

All of the parameters are optional and if they are not specified inputs from \texttt{inputs.conf} are used. 
The two standard input files for the ANITA-III and ANITA-IV flights come with the package.

The output directory contains a series of root files with  information about all the neutrinos simulated, as well as a text file containing the neutrino survival efficiency at different stages of \icemc, and the volumetric acceptance.

Other programs test a portion of the full \icemc program:
\begin{itemize}
    \item {\tt testThermalNoise} simulates only the thermal noise at the payload for a specific ANITA flight.
    \item {\tt testInputAfterAntenna} simulates the injection of an RF impulse after the antenna feed; this program is used to produce trigger efficiency scans similar to the ones taken before each ANITA flight.
    \item {\tt testWAIS} simulates the WAIS pulser as described in Subsection~\ref{subsec:wais}.
\end{itemize}

To produce ANITA-like output files and use more advanced features of \icemc, the installation of \texttt{libRootFFTWwrapper} (\url{https://github.com/nichol77/libRootFftwWrapper/}) and the ANITA \texttt{eventReaderRoot} (\url{https://github.com/anitaNeutrino/eventReaderRoot}) is necessary. 

