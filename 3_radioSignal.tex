\section{Radio signal simulation} 
\label{sec:rf}

\subsection{Askaryan parametrization}
For each event the neutrino energy is picked randomly following the input energy spectrum.
Rather than simulating the particle shower development in the ice that
would produce the Askaryan radiation, a 1D parametrization is used 
to find the peak of the Askaryan signal at 1\,m from the point of interaction, $\mathcal{E}^{(\mathrm{@ 1m})}$, 
according to Reference~\cite{JaimeAskarian2000} and following:
\begin{equation}
\label{eq:vmmhz}
\mathcal{E}^{(\mathrm{@ 1m})} =2.53\times 10^{-7}\cdot \frac{\sin{\theta_{\mathrm{view}}}}{\sin{\theta_{\mathrm{Ch}}}}\cdot  \frac{E}{\mathrm{TeV}} \cdot \frac{\nu}{\nu_0} \cdot \frac{1}{1+\left( \frac{\nu}{\nu_0} \right)^{1.44}} \;,
\end{equation}
\noindent where $\nu$ is the frequency, $\nu_0=1.15$\,GHz, $E$ is
the shower energy (see Subsection~\ref{subsec:emhadshower}),
%\CD{Is the hadronic shower treated differently from the em?}
$\theta_{\mathrm{view}}$ is the viewing angle and $\theta_{\mathrm{Ch}}$ is the Cherenkov angle.
This parametrization is valid up to 5\,GHz, covering the ANITA
band of 0.2-1.2\,GHz.

\subsection{Neutrino flavors and interaction types}
\label{subsec:emhadshower}
The \icemc program assumes flavor democracy, assuming 
the flavors are fully mixed before the neutrinos get to the ice.
Information about each flavor is stored separately so the sensitivity to each flavor may be quoted separately.

The neutrino event undergoes a charged/neutral current interaction in
$68.7\%$/31.3\% of cases.
%of the events at random are chosen to be charged current and the remainder to be neutral current.
For charged current electron neutrino interactions, the electromagnetic
component is calculated as $1-y$, where $y$ is the inelasticity and is
approximated by a double-exponential function~\cite{gandhi}. 
In all other cases the electromagnetic component is considered negligible.
The hadronic component is equal to the inelasticity.
Secondary particles are not propagated.

%A future version of the code will include $\mu/\tau$ bremsstrahlung, photo-nuclear interactions, and $\tau$ neutrino regeneration.

\subsection{Cherenkov Cone}
\label{subsec:cherenkov_width}
The width of the Cherenkov cone is parametrized separately for the electromagnetic and hadronic components of the shower, according to References \cite{JaimeAskarian2000} and \cite{jaime05}, respectively.
The width of the electromagnetic
component in degrees is characterized by:
\begin{equation}
\label{eq:deltheta_em}
\Delta\theta_{em}(\nu)=2.7^{\circ} \cdot \frac{\nu_0}{\nu}\cdot \left(
  \frac{E_{LPM}}{ 0.14 E_{EM}+E_{LPM}} \right)^{0.3} \;,
\end{equation}
where 
$E_{EM}$ is the part of the shower energy associated with electromagnetic particles, and 
$E_{LPM}$ is the energy above which the Landau-Pomeranchuk-Migdal
(LPM) effect becomes important.  
The LPM effect causes
the bremsstrahlung interaction to become suppressed when the longitudinal
momentum transfer ($\propto k/E^2$) becomes so small 
that the Heisenberg uncertainty
causes the interaction to occur over many
scattering centers, resulting in destructive interference.  
This effect reduces the width of the Cherenkov cone, but not the magnitude of the electric field at the Cherenkov angle.
Following the recommendation in Reference~\cite{JaimeAskarian2000}, 
$E_{LPM}$ is set to $2\times10^{15}\ev$ in ice,
and can be scaled to other media using the ratio of the respective radiation lengths.
%~\cite{pdg}:
%\begin{equation}
%E_{LPM}=2\times10^{15} \frac{X_0^{\mathrm{medium}}}{X_0^{\mathrm{ice}}}
%\end{equation}
%where ${X_0^\mathrm{ice}}=39.5$\,cm is the radiation length of ice and
%$X_0^{\mathrm{medium}}$ is the radiation length of the medium of interest.


The width of the Cherenkov cone for hadronic showers is modeled as laid out in Equation 9
in~\cite{jaime05}:
\begin{equation}
\Delta \theta_{\mathrm{had}} (\nu) =\frac{c}{\nu} \cdot
	\frac{\rho}{\mathrm{K}_{\Delta} X_0} \cdot
	\frac{1}{n^2-1} \;,
\end{equation}
\noindent where
$c$ is the speed of light,
$\nu$ is the frequency considered, 
$\rho$ is the density of ice,
$\mathrm{K}_{\Delta}$ is a normalization constant determined by a
separate Monte Carlo simulation~\cite{jaime05},
$X_0$ is the radiation length, and
$n$ is the index of refraction.

The signal strength at viewing angle $\theta_{\mathrm{view}}$ 
away from the Cherenkov angle $\theta_{\mathrm{Ch}}$ is also parametrized following
Equation\,13 in Reference\,~\cite{jaime05}:
\begin{equation}
\mathcal{E}(\theta_{\mathrm{view}})=\frac{\sin{\theta_{\mathrm{view}}}}{\sin{\theta_{\mathrm{Ch}}}} \cdot
\mathcal{E}(\theta_{\mathrm{Ch}})\cdot \exp\left[-\left(
    \frac{\theta_{\mathrm{view}}-\theta_{\mathrm{Ch}}}{\Delta\theta_{\mathrm{em,had}}} \right)^2
\right] \;.
\end{equation}




