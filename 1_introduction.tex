\section{Introduction}
The ANtarctic Impulsive Transient Antenna (ANITA) experiment is a NASA-funded, balloon-borne experiment aiming primarily to detect Ultra-High-Energy (UHE) neutrinos and cosmic rays~\cite{ANITA1detector,ANITA1paper,ANITA2paper,ANITA2erratum}.
UHE neutrinos may be produced when cosmic rays interact with the
cosmic microwave background, either through the Greisen Zatsepin
Kuzmin (GZK)~\cite{greisen1966end,zatsepin1966gt} mechanism or through photo-disintegration, or directly from an astrophysical source.
UHE neutrinos interact in the Antarctic ice and produce a coherent radio pulse of Askaryan radiation~\cite{askaryan}. UHE cosmic rays interact with the geomagnetic field to produce a coherent radio impulse~\cite{suprun2003synchrotron,falcke2003detecting,ANITA1UHECR}. The geometry of the Earth's magnetic field in Antarctica allows ANITA to distinguish the signatures of these two types of events through their polarization.

The ANITA experiment flies approximately 40\,km above Antarctica looking for radio signals in the band between 200 and 1200\,MHz, produced by UHE neutrinos and cosmic rays.
Figure~\ref{fig:intro_ANITAconcept} shows a basic scheme of the
Askaryan neutrino detection at the ANITA experiment, as well as the UHE cosmic ray detection principle.
Four ANITA flights (Section~\ref{sec:anita3}) have been successfully
completed as of 2019, and this paper focuses on the simulations of the third and fourth ANITA flights.

\begin{figure}[!h]\centering
  \includegraphics[width=.8\linewidth]{./Figs/ANITA_scheme_icemcpaper.png}
  \caption{The ANITA detection concept with a photo of the ANITA-IV
    payload. UHE neutrinos interact with the Antarctic ice and produce
    a coherent radio pulse of Askaryan radiation. UHE cosmic ray
    interactions in the atmosphere produce a shower of secondary
    particles that interact with the geomagnetic field and can also
    produce a coherent radio impulse. Both these signals are detected by the ANITA instrument. 
  }
  \label{fig:intro_ANITAconcept}
\end{figure}



The \icemc program is a C++ Monte Carlo simulation program based on
ROOT~\cite{brun1997root} used to simulate the Askaryan radiation produced
by neutrino interactions and the response of the ANITA detector to this
radiation.
This program is used by the ANITA collaboration to tune the selection
cuts of the cosmogenic neutrino analysis and quantify the experiment's sensitivity.


\begin{figure}[!h]\centering
  \includegraphics[width=.8\linewidth]{./Figs/IcemcFlowchart.png}
  \caption{Flowchart of the \icemc simulation for a single candidate.}
  \label{fig:intro_icemcFlow}
\end{figure}

A flowchart of the \icemc simulation steps is shown in Figure~\ref{fig:intro_icemcFlow}.
At the beginning of each run the user can choose which ANITA flight to simulate and which neutrino energy spectrum to use.
For each neutrino event, the position of the payload along the ANITA flight path, as well as the neutrino interaction position are randomly chosen.
These are used to find the path to the ANITA detector (Section~\ref{sec:eventGeometry}). 
The neutrino direction is then chosen within detectable angles; the neutrino energy is chosen following the input theoretical model; and the neutrino flavour and interaction type are chosen according to their expected ratios. 
These parameters are used to produce the radio-frequency pulse following the Askaryan model described in Section~\ref{sec:rf}.
The signal is then propagated through the ice and air to the ANITA
detector (see Section~\ref{sec:propagation}). 
The response of the ANITA signal chain is simulated, and the resulting 
data are saved in the same format as ANITA flight data 
(see Section~\ref{sec:ANITA}).
Different parts of the simulation are validated against data taken
both at the NASA Long Duration Balloon Facility
before the launch and in-flight during the past ANITA flights (see
Section~\ref{sec:validation}).
Finally the neutrino acceptance of the third and fourth ANITA flights, and
the contributions of the different parts of the simulation to the uncertainties on the acceptance are presented in Section~\ref{sec:results} .


\section{The ANITA flights}
\label{sec:anita3}
The first two ANITA payloads flew in 2006-2007\cite{ANITA1paper} and 2008-2009\cite{ANITA2paper,ANITA2erratum}, respectively.
Although \icemc can be used to simulate older flights, this paper focuses on the simulation of the third and fourth flights.

Figure~\ref{fig:ANITA_flightPath} shows the flight paths of the ANITA-III and ANITA-IV instruments. 
The color map shows the Antarctic ice depth. 
The third ANITA payload (ANITA-III) launched on December 18$^{\text{th}}$, 2014 from the
NASA Long Duration Balloon (LDB) facility near McMurdo Station, Antarctica.
ANITA-III followed the polar vortex flying at the altitude of 37\,km for
22 days until January 9$^{\text{th}}$, 2015 when the flight was terminated
near the Australian Davis Station.
Similarly, the fourth ANITA payload (ANITA-IV) launched on December 2$^{\text{nd}}$, 2016 from the NASA LDB facility in Antarctica and landed approximately 100\,km from the South Pole Station on December 29$^{\text{th}}$, 2016.


\begin{figure}[!h]\centering
  \includegraphics[width=.45\linewidth]{./Figs/Flightpath_ANITA3.png}
  \includegraphics[width=.45\linewidth]{./Figs/Flightpath_ANITA4.png}
  \caption{Flight path simulated in \icemc for the ANITA-III (left) and ANITA-IV (right) flights.
    Antarctica map produced using the BEDMAP model~\cite{bedmap}.     }
  \label{fig:ANITA_flightPath}
\end{figure}



\begin{figure}[!h]\centering
  \includegraphics[width=.95\linewidth]{./Figs/ANITA3_signalChain1.png}
  \includegraphics[width=.95\linewidth]{./Figs/ANITA3_signalChain2.png}
  \caption{ANITA-III signal chain.
  The receiving antenna splits signals into vertical and horizontal
  polarizations which follow identical paths. 
After going through filters and amplifiers, the signal is split into
the trigger and digitizer paths. In the trigger path the signal passes
through a tunnel diode and is compared to a threshold. 
When a trigger is issued, a switched capacitor array digitizes the
signal with a mean sample rate of 2.6\,GSa/s.}
  \label{fig:ANITA3_signalChain}
\end{figure}


The ANITA-III and ANITA-IV instruments were similar to the two previous
ones~\cite{ANITA1paper,ANITA2paper}, using 48 dual-polarization quad-ridged horn antennas with bandwidth from 200 to 1200\,MHz. 
The antennas were arranged in three rings forming 16 azimuthal sectors per ring, with the top ring divided into two layers to fit the launch envelope specifications, as shown in Figure~\ref{fig:intro_ANITAconcept}.

Figure~\ref{fig:ANITA3_signalChain} shows a schematic of the ANITA-III signal chain. 
The receiving antenna measures vertical and horizontal polarization components of incident radio waves, which then follow twin paths. 
After passing through filters and amplifiers, the signal is split into
the trigger and digitizer paths.
In the trigger path the signal passes
through a tunnel diode which acts as a square-law detector over the
input. The output of the tunnel diode is compared to the channel
threshold, which is dynamically adjusted by a PID loop to maintain the
rate for each antenna around 500\,kHz. A combination of first-level (L1) and second-level (L2) triggers forms a
global trigger which can be issued in each polarization.
When a trigger is issued, a switched capacitor array digitizes the signal with a mean sample rate of 2.6\,GSa/s. 
The ANITA-IV signal chain included the Tunable Universal Filter Frontend boards (see Subsection~\ref{subsec:tuffs}), and $90^{\circ}$ hybrids to convert the signals into left- and right- circularly polarized (LCP and RCP) components. 
The ANITA-IV trigger logic also required a coincidence between LCP and RCP signals within a 4\,ns coincidence window to select for linearly polarized signals. 

As the payload rotates freely, two sets of GPS units were used
independently to determine the
payload position and attitude.
Power was supplied by an octagonal array of photovoltaic panels and stored using four pairs of 12-V lead-acid batteries.
Communication to and from the payload was through the Iridium and TDRSS satellite systems throughout the flight, and by a direct line-of-sight radio link when within range of McMurdo Station.




